\documentclass{article}
\usepackage{amsmath, amssymb, amsthm}

% Define the theorem environment for the proof
\newtheorem{theorem}{Theorem}
\newtheorem{lemma}{Lemma}
\newtheorem{definition}{Definition}

\begin{document}

% Title and author (optional)
\title{Normal 2 Player Burning Game Proof of a game}
\author{Isaac Gilbert}
\date{} % Remove date if not needed
\maketitle

\begin{definition}
A normal 2 player burn graph denoted by \(BG\). Let G be a graph, \(G=(V,E)\). 
Each Vertex can be White, Blue, Red or Purple
\\\\Each turn has 2 steps:
\\\\Step 1, let player 1 pick \(v\) then let player 2 pick \(u\) where \( v,u \in V_{white} \). Set \(v\) to Red and \(u\) Blue.
\\\\Step 2, For all \(v\) where \( v\in V_{white} \), let \({u_1,...,u_n}\) be all the adjacent to \(v\) where \(n\) is the number of adjacent vertexs.\\
If \((\exists i \in {1,...n} | u_i \text{ is Purple } )\vee (\exists i,j \in {1,...n} | u_i \text{ is Blue } \wedge u_i \text{ is Red })\) then set \(v\) to Purple.\\ 
If \((\forall i \in {1,...n} | u_i \text{ not is Purple }) \vee (\exists i \in {1,...n} \wedge \forall j \in {1,...n} | u_i \text{ is not Blue } \wedge u_j \text{ is Red })\) then set \(v\) to Red.\\ 
If \((\forall i \in {1,...n} | u_i \text{ not is Purple }) \vee (\exists j \in {1,...n} \wedge \forall i \in {1,...n} | u_i \text{ is Blue } \wedge u_j \text{ is not Red })\) then set \(v\) to Blue.\\ 
If \((\forall i \in {1,...n} | u_i \text{ not is Purple }) \vee (\exists i,j \in {1,...n} {1,...n} | u_i \text{ is not Blue } \wedge u_j \text{ is not Red })\) then remain \(v\) to White.\\ 

The game ends when there does not exists any White vertexs.

If the total number of Red vertexs is greater than the total number of Blue vertexs then it is Red Wins
If the total number of Red vertexs is less than the total number of Blue vertexs then it is Blue Wins
If the total number of Red vertexs is equal than the total number of Blue vertexs then it is Tie
\end{definition}

\begin{theorem}
The maximum length of a game is $\lceil \frac{n}{3} \rceil$ where n is the number of Vertex.
\end{theorem}

\begin{proof}
Lets be given a connected graph using the definition of \(BG\). While the set of white vertex $\le 3$ 
then every turn both player burning 1 vertexs in step 1, let denote them be $v, u$. Hence we need to show that at least 1 is burn in step 2.
Since burning graphs are on connected graphs then each vertex must be at least 1 path to every other vertex.
Select the vertex $v$ and select any white vertex and then make a path between them. The vertex on this path will be denote by ${w_1, w_2,..., w_m}$
where $m$ is the length of the path and $w_m$ is the orginal white vertex.\\\\
If all $w \in {w_1, w_2,..., w_{m-1}}$ are white then $v$ which is adjacent to $w_1$ burns it.
If all $w \in {w_1, w_2,..., w_{m-1}}$ are none-white then $w_{m-1}$ which is adjacent to $w_m$ burns it.
If there exists $w \in {w_1, w_2,..., w_{m-1}}$ which is none-white and there exists $w \in {w_1, w_2,..., w_{m-1}}$ which is white,
then select the first white vertex on the path, denote it as $w_i$ which is adjacent to $w_{i-1}$ so $w_i$ will be burnt.
Hence while there is greater than or equal to 3 white vertex, then at least 3 vertex burn a turn.
\end{proof}

\begin{definition}
A graph denoted with $G(V,E)$ is symmetric if for any two pairs of adjacent vertices $(u,v)$ and $(w,x)$ where $u,v,w,x \in V$, 
there exists a bijection from the vertex set to itself that preserves adjacency that maps $(u,v)$ to  $(w,x)$.
\end{definition}

\begin{theorem}
On any symetric graph in a 2-Player Burning Graph, player 2 can deploy a strategy to always tie.
\end{theorem}

\begin{proof}
The aim is to show there exist a strategy guarentees to burn the same amount in step 1 and step 2 of the game.
Let $g$ be a symetric graph denoted by $G(V,E)$. Let player 1 select $v$ that is adjacent to $u$, where $v,u \in V$. 
As player 2 select $v'$ that is adjacent to $u'$ where $(v',u')$ uses the bijection that exists from sysmetics graph definition
that there is mapping from $(v',u')$ to $(v,u)$. Since this preserves adjancey when player 1 burns a vertex then player 2 burns the equilecent vertexs through the bijection in step 2.
Since there must be an even amount of vertex and $2x$ are burnt a turn where $x$ is the vertex player 1 burns.
Therefore in step 1 the can not be a single vertex left.
\end{proof}

\begin{theorem}
BG is PSPACE
\end{theorem}

\begin{proof}
Let G be a graph with n vertices, using Theorem 1 hence the maximum number of turn is $\lceil \frac{n}{3} \rceil$ and each turn only 1 is select by each player limited to 1 colour, then it can be solved using a polynomial amount of space.
\end{proof}

\end{document}